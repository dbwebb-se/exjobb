%
% Project plan for degree project (TE2502)
% Template version: Don't forget to update when updating the template
\newcommand{\theVersion}{1.4 -- April 28, 2019}
%
\documentclass[12pt,a4paper,twoside]{article}
\usepackage{times}
\usepackage{multirow}
\usepackage{hyperref}
\usepackage[utf8]{inputenc}
\usepackage[T1]{fontenc}
\usepackage[top=2.5cm,bottom=2.5cm,left=2.5cm,right=2.5cm]{geometry}
% --------------------------------------------
% package "todonotes" is used for notes and comments
% To disable all notes you can use the option "disable", see second row below:
\usepackage[color=blue!10,textsize=footnotesize,textwidth=25mm]{todonotes}
%\usepackage[disable]{todonotes} %passive=do not show
% --------------------------------------------
\usepackage[sort&compress]{natbib}
\usepackage{enumitem}
\setlist{topsep=1ex,itemsep=0.5ex,parsep=0pt,partopsep=0pt}  % this sets the vertical spacing between list items and surrounding parapgraphs
\setlength{\bibsep}{4pt}

% Redefine the course code and name.
% PA1438: Självständigt arbete i Webbprogrammering
% PA1438: Diploma project in Web Programming
\newcommand{\theCourse}{PA1438: Diploma project in Web Programming}

% Choose one of the following for the purpose of the submission.
\newcommand{\thePurpose}{Thesis Topic} % Use for a project idea
%\newcommand{\thePurpose}{Project plan} % Use for a project plan

% Redefine the company name as a variable for easier reference.
\newcommand{\theCompany}{CompanyOrganizationName}

% *** Do not touch the following lines. BEGIN. ***
\title{\thePurpose\ for diploma project\\\vspace{1mm}\small{Version \theVersion}}
\author{\textsc{\theCourse}}
\date{\today}
\begin{document}
\maketitle
\vspace*{-5mm}
% *** END. Do not touch the lines above. ***

% *** PLEASE START HERE ***
\noindent % Do not delete this line or add an empty line below.
\begin{tabular}{|l|l|p{10.4cm}|}
\hline
Title      		  & \multicolumn{2}{|p{13cm}|}{Provide a concise, specific and informative title} \\
\hline
Area      		  & \multicolumn{2}{|l|}{E.g., Industrial Economics} \\
\hline\hline
\multirow{4}{*}{Student 1}
 & Name           & Your full name as given in Ladok \\\cline{2-3}
 & E-Mail         & ...@student.bth.se \\\cline{2-3}
 & Person nr      & YYMMDDXXXX \\\cline{2-3}
 & Program        & Program you are registered in \\\cline{2-3}
\hline
\multirow{4}{*}{Student 2}
 & Name           & Leave blank if you work alone \\\cline{2-3}
 & E-Mail         &  \\\cline{2-3}
 & Person nr      &  \\\cline{2-3}
 & Program        &  \\\cline{2-3}
\hline
\multirow{3}{*}{Supervisor}
 & Name \& title  & Leave blank if no supervisor is assigned. You might ask for a particular supervisor here. \\\cline{2-3}
 & E-Mail         &  \\\cline{2-3}
 & Department     &  \\
\hline
\multirow{3}{*}{Co-advisor}
 & Name \& title  & Contact person at company or organization \\\cline{2-3}
 & E-Mail/~phone  &  \\\cline{2-3}
 & Company        & \theCompany \\
\hline
\end{tabular}

\todo[inline]{Don't forget to delete the blue comment-boxes before submission.
See \textbackslash usepackage-rows in the LaTeX-code.}
\todo[inline]{You need to include and cite some references that are relevant for your work,
at least in Section \ref{sec:intro} and Section \ref{sec:method}. This can include references to important and/or recent works in the area regarding key concepts and terminology, related work, research designs, etc. You do not need to cite the course literature \cite{berndtsson2007thesis,evans2014write,blomkvist2014metod,host2006att,trochim2015research}. These are added here just to show you how the citation marks and references should look like. Note that references are numbered in alphabetic order. More information about academic writing can be found in Evans et al.'s book \cite{evans2014write}.}


\section{Introduction}
\label{sec:intro}
In the introduction, you should describe the field you are working in and eventually lead the reader to the specific problem that you intend to address in your thesis. It is important that your presentation is based on established knowledge and that it clearly identifies a relevant problem that is of general interest.

Typically, an introduction proceeds from the general to the specific. As a rough ``guideline'', you can think of the following things that an introduction should do (not necessarily in the exact order described below).
\begin{enumerate}
    \item Establish the importance of the (general) field.
    \item Provide relevant background facts or information.
    \item Define key terminology (if required).
    \item Present the problem area or research focus, i.e. moving from the general field (see 1.) to the specific area you are dealing with.
    \item Give a brief overview over existing research/~contributions in the specific area you are dealing with.
    \item Identify a gap in this work.
    \item Describe the specific problem you will address (as a result from 6.).
    \item Explain in which ways the problem (as described in 7.) is relevant.
\end{enumerate}
A sentence or two for each item should suffice for a project idea.


\subsection{Ethical, societal and sustainability aspects}
\label{sec:ethics}
In the project plan, you should briefly discuss aspects regarding ethics, society and sustainability. If you think there are no such aspects that are relevant for
your work, please argue why this is the case.

Please note that subsections should be formatted and numbered like this one.
The paragraphs of sections, subsections and subsections are not indented.


\subsubsection{Example sub-subsection}
You can introduce further sections, subsections or sub-subsections. Subsections beyond level 3 should be avoided.


\section{Aim, objectives and thesis question(s)}
\label{sec:aim}
This section should contain concise and clear descriptions of your overall aim (goal),
how that is broken down into more specific objectives (sub-goals) and how those are
operationalized into answerable (research) questions and/or testable hypotheses.
Please make sure that you will actually reach your overall goal by answering your questions.


\section{Method}
\label{sec:method}
This section should describe the research design or methods you intend to use to gather the evidence (data) you need to solve the problem or answer the research questions, and explain why these
methods are appropriate. If you intend to measure certain properties or attributes, you
need to explain why these properties or attributes are relevant and how they could be measured
in a reliable way.

Please make sure to discuss whether there are aspects in your work that might require
approval by an ethics committee (see Section \ref{sec:ethics}). This is particularly important
when your research involves experiments with humans.


\section{Expected outcomes}
\label{sec:outcome}
This section should briefly describe what you expect as an outcome and in which ways this outcome is important/helpful and for whom.


\section{Time and activity plan}
\label{sec:plan}
This section should provide information about project planning and project tracking.
Project planning means a list of tasks that will be carried out to reach your aim,
and how these tasks are placed in time, i.e. a \textit{work breakdown structure}
and a \textit{schedule}.

Please include activities beyond just preparing and submitting the required documents.
A trivial work breakdown structure like ``planning, data collection, analysis,
report writing'' is not sufficient. You don't need to prepare a detailed weekly
schedule, but you must try to be realistic. Make sure that you don't forget to
schedule some time for unexpected delays.

Project tracking means a description of how you intend to follow up you work and
make it possible to check whether you are ``on track''. This should, among others,
include a brief description of how you and your supervisor plan to organize
the supervision.


\section{Limitations and risk management}
\label{sec:risks}
This section should provide a description of the limitations of your
work, and a brief analysis of \textit{risks}, i.e. what might go wrong. For each risk that
is either serious or likely to happen, you should have a \textit{mitigation plan}, i.e.
what you do to prevent risks and what do you do when they actually happen.
Make sure to think about non-trivial risks.


% All references are stored in a separate bib-file: thesis-refs.bib
\bibliography{thesis-refs}
\bibliographystyle{plain}


% ``Agreement'' with company/ organization plus signatures
% Only needed for projects in collaboration with external companies or organizations
% Comment out when not needed
\appendix
\newpage
%\section{Signaturbilaga}
Företaget/~organisationen \textit{\theCompany} godtar den beskrivning av ``uppdraget'' i
detta dokument (``Idéskiss'').
\textit{\theCompany} åtar sig också att i samband med det framtida examensarbetet ge
studenten/~studenterna en rimlig handledning i samband med dennes/~deras utförande av
examensarbetet i de delar som är relaterade till ``uppdraget'' på \textit{\theCompany}.
\textit{\theCompany} är medvetet om att BTH inte ingår något avtal med företaget och
heller inte garanterar att ``uppdraget'' kan lösas på ett tillfredsställande sätt.

Den delen av examensarbetet som görs av studenten/~studenterna på ``uppdraget'' av
\textit{\theCompany} är enbart en relation mellan studenten/~studenterna och företaget/~organisationen.
BTH kommer att bedöma examensarbetet i form av rapporter som skickas in för bedömning.
Dessa rapporter är offentliga handlingar.
Den godkända slutrapporten kommer att publiceras i
dokumenthanteringssystemet ``DIVA'' och blir då publik och allmän handling i samband
med denna publicering.

Denna signaturbilaga är inget juridiskt avtal och därför inte bindande för någon part
(BTH eller \textit{\theCompany}) utan skall endast betraktas som en ``intentionsförklaring''
från \textit{\theCompany}.

\vspace{12mm}
Karlskrona, \today
\vspace{12mm}

\rule{10cm}{1pt}

Signatur, handledare på \theCompany
\vspace{12mm}

\rule{10cm}{1pt}

Namnförtydligande
 % På svenska
\section{Signature enclosure}
The company/~organization \textit{\theCompany} accepts the description of the ``work'' in
this document (``Project idea'').
\textit{\theCompany} also accepts to offer
the student/~students a reasonable amount of supervision in connection with the future degree project in relation to the parts of the degree project that are related to the ”work” at \textit{\theCompany}.
\textit{\theCompany} is aware that BTH is not part of any agreement with the company and does not guarantee that the ``work'' is carried out in a satisfactory manner.

The part of the degree project that is carried out by the student/~students ``on behalf of''
\textit{\theCompany} is only a commitment between the student/~students and the company.
BTH will evaluate the work in the form of documents that are submitted for evaluation.
All submitted works can be made public on request. 
According to regulations, the passed final report will be published in the document management system ``DIVA'' and as a part of this publication will then become a generally accessible public document.

This signature enclosure is not a legal agreement and is therefore not binding for any part
(BTH nor \textit{\theCompany}) but should be considered as a ``declaration of intention''
from \textit{\theCompany}.

\vspace{12mm}
Karlskrona, \today
\vspace{12mm}

\rule{10cm}{1pt}

Signature by supervisor at \theCompany
\vspace{12mm}

\rule{10cm}{1pt}

Printed name
 % In English

\end{document}
