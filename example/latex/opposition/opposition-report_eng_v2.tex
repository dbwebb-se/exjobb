%
% Title: Opposition of degree project
% Template version: See \theVersion below  %% Don't forget to update.
%
\documentclass[12pt,a4paper,twoside]{article}
\usepackage{times}
\usepackage{multirow}
\usepackage{hyperref}
\usepackage[T1]{fontenc}
\usepackage[utf8]{inputenc}
\usepackage[top=2.5cm, bottom=2.5cm, left=2.5cm, right=2.5cm]{geometry}
\usepackage[color=blue!10,textsize=footnotesize,textwidth=25mm]{todonotes}
%
% --- PLEASE DO NOT CHANGE THE LINES BELOW ---
% Do only change the version number below when changing the template.
\newcommand{\theVersion}{2.0 -- April 18, 2018}
%
% Please change the course name, if necessary.
% PA1438: Självständigt arbete i Webbprogrammering
% PA1438: Diploma project in Web Programming
\newcommand{\theCourse}{PA1438: Diploma project in Web Programming}
%
% Do not touch the following 3 lines.
\title{Opposition of diploma project\\\vspace{0.4ex}\small{Version \theVersion}}
\author{\textsc{\theCourse}}
\date{\today}  % This will automatically insert the current date
%
\begin{document}
%
\maketitle
%
% --- PLEASE INSERT YOUR REVIEW BELOW ---
%
\vspace*{-3ex}
\noindent
\begin{tabular}{|l|l|p{10cm}|}
\hline
\multirow{3}{*}{Opponent}
 & Name                  & Your full name as given in LADOK  \\\cline{2-3}
 & e-Mail                & ...@student.bth.se \\\cline{2-3}
 & Social security nr    & YYMMDDXXXX \\\hline
\multirow{2}{*}{Thesis}
 & Title                 &  \\\cline{2-3}
 & Author(s)             &  \\\hline
\end{tabular}

\todo[inline]{This structure is just a suggestion. Please adapt it as you see fit. A typical opposition should be 2000--3000 words in size (5--8 pages in 12pt font with single line spacing).

Make sure to be as specific as possible in your review and change requests and describe clearly what should/might be improved; where, why and how.

Don't forget to comment out the blue boxes before you submit your review.}

\section{Introduction}
\todo[inline]{This section should provide the following: (a) a brief overview over the reviewed work,
(b) resources you used for your review, if applicable (e.g., reviewing guidelines), and (c) a brief summary of the main strength and weaknesses of the reviewed work.}


\section{Critical review}
\todo[inline]{In this section, you provide more details about the strengths and weaknesses. It typically works well, if the order follows the structure of the reviewed work, but any other structure that you seem fit is fine. If you include suggestions for change here, make sure to also list them in Section \textit{Required changes} or \textit{Recommended changes}.}


\section{Required changes}
\todo[inline]{In this section, you should list all changes that you think are necessary before publication.
Please make sure the following: the places for a changes are easy to find, (b) there is a motivation for the change (if it is not obvious), and (c) you give suggestions for improvement (if it is not obvious how the problem can be resolved).}


\section{Recommended changes}
\todo[inline]{In this section, you can list further change requests that are less critical than the ones in Section \textit{Required changes}.}


\section*{References}
\todo[inline]{You can skip this section (including the header), if you do not have any references.}

\end{document}
